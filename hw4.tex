\documentclass[11pt]{article}
\usepackage{amsthm}
\usepackage{xspace}
\usepackage{siunitx}
\usepackage{booktabs}
\usepackage{miscdoc,multirow,bigstrut,bigdelim,colortbl}
\usepackage{setspace}
\usepackage{listings}
\usepackage{graphicx}
\usepackage{amsmath}
\usepackage{subfig}
\usepackage{intcalc}
\usepackage{multirow}
\usepackage{polynomial}
\usepackage{polynom}


\DeclareUnicodeCharacter{2212}{-}
\begin{document}
\title{Homework-4}
\author{Mustafa Tokat}
\maketitle
\section{Introduction}

\paragraph{}\section{Analysis of Problems}
\subsection{Problem 1} \textbf{Consider the prime p = 9929 and the primitive element 2.}
\subsubsection{\textbf{Show the steps of the Diffie-Hellman between Alice and Bob for a = 1983 and b = 2014.}}
% Table generated by Excel2LaTeX from sheet 'Sayfa1'
% Table generated by Excel2LaTeX from sheet 'Sayfa1'
\begin{table}[htbp]
    \centering
    \caption{Add caption}
      \begin{tabular}{|c|c|c|}
      \toprule
      Alice & (p, q) = (9929, 2) & Bob \\
      \midrule
      a=1983 &       & b=2014 \\
  \cmidrule{1-1}\cmidrule{3-3}    \multirow{2}[2]{*}{} &       & \multirow{2}[2]{*}{} \\
            &       &  \\
  \cmidrule{1-1}\cmidrule{3-3}    \multicolumn{1}{|c|}{\multirow{2}[2]{*}{2**1983 (mod 9929)                                       = 8580}} &       & \multicolumn{1}{c|}{\multirow{2}[2]{*}{2**2014 (mod 9929)              = 5387}} \\
            &       &  \\
  \cmidrule{1-1}\cmidrule{3-3}    \multirow{2}[2]{*}{} &       & \multirow{2}[2]{*}{} \\
            &       &  \\
  \cmidrule{1-1}\cmidrule{3-3}    5387**1983(mod 9929) &       & 8580**2014(mod 9929) \\
  \cmidrule{1-1}\cmidrule{3-3}    \multirow{2}[2]{*}{} &       & \multirow{2}[2]{*}{} \\
            &       &  \\
  \cmidrule{1-1}\cmidrule{3-3}    K = 7690 &       & K = 7690 \\
  \cmidrule{1-1}\cmidrule{3-3}    \end{tabular}%
    \label{tab:addlabel}%
  \end{table}%
  
  
\newpage\subsubsection{\textbf{What is the value of the agreed secret key?}}

\textbf{Result: }7690

\subsection{Problem 2} \textbf{Consider the RSA public and private key pairs: (e, n) = (17, 902801) and (d, n, p, q, $\phi $ ) =
(423953, 902801, 911, 991, 900900).}\\
 
I have checked to all values(indeed, to practice). 

\subsubsection{\textbf{Given $M_{1}$ = 500000, compute $C_{1}$ = $M_{1}^{e}$ (mod n).}}

Let's compute according to above  formula:

$C_{1}$ = $500000^{17}$ (mod 902801) = 487730

\subsubsection{\textbf{Given $C_{2}$ = 707631, compute $M_{2}$ = $C_{2}^{d}$  (mod n)}}

Similarly; 

$M_{2}$ = $707631^{423953}$ (mod 902801) = 500001

\subsection{Problem 3} \textbf{RSA with three primes would also work: n = pqr, $\phi $(n) = (p−1)(q−1)(r−1), gcd(e, $\phi $(n)) = 1,
and d = e −1 (mod $\phi $(n))}

\subsubsection{\textbf{Setup an example RSA public/private key pair using primes 29, 31, 37, and e = 17.}}

\subsubsection{\textbf{Encrypt m = 10000 and then decrypt the ciphertext.}}

\subsubsection{\textbf{Explain why RSA with three primes algorithm is not preferred.}}


\end{document}
