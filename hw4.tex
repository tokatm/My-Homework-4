\documentclass[11pt]{article}
\usepackage{amsthm}
\usepackage{xspace}
\usepackage{siunitx}
\usepackage{booktabs}
\usepackage{miscdoc,multirow,bigstrut,bigdelim,colortbl}
\usepackage{setspace}
\usepackage{listings}
\usepackage{graphicx}
\usepackage{amsmath}
\usepackage{subfig}
\usepackage{intcalc}
\usepackage{polynomial}
\usepackage{polynom}


\DeclareUnicodeCharacter{2212}{-}
\begin{document}
\title{Homework-4}
\author{Mustafa Tokat}
\maketitle
\section{Introduction}

\paragraph{}\section{Analysis of Problems}
\subsection{Problem 1} \textbf{Consider the prime p = 9929 and the primitive element 2.}
\subsubsection{\textbf{Show the steps of the Diffie-Hellman between Alice and Bob for a = 1983 and b = 2014.}}
\subsubsection{\textbf{What is the value of the agreed secret key?}}

\subsection{Problem 2} \textbf{Consider the RSA public and private key pairs: (e, n) = (17, 902801) and (d, n, p, q, $\phi $ ) =
(423953, 902801, 911, 991, 900900).}
\subsubsection{\textbf{Given M1 = 500000, compute C1 = Me 1 (mod n).}}
\subsubsection{\textbf{Given C2 = 707631, compute M2 = Cd 2 (mod n)}}


\subsection{Problem 3} \textbf{RSA with three primes would also work: n = pqr, $\phi $(n) = (p−1)(q−1)(r−1), gcd(e, $\phi $(n)) = 1,
and d = e −1 (mod $\phi $(n))}
\subsubsection{\textbf{Setup an example RSA public/private key pair using primes 29, 31, 37, and e = 17.}}
\subsubsection{\textbf{Encrypt m = 10000 and then decrypt the ciphertext.}}
\subsubsection{\textbf{Explain why RSA with three primes algorithm is not preferred.}}


\end{document}
